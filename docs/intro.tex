\chapter{Bevezetés}
\label{ch:intro}

\section{Motiváció}

A szakdolgozatom az implicit felületek előjeles távolságfüggvényeinek különböző módszerekkel történő megjelenítésével, valamint a módszerek összehasonlításával foglalkozik. A témával már korábban egy grafika beadandó formájában is foglalkoztam, ahol implementáltam a naiv, relaxált és javított gömbkövetési módszereket, ezeket egy negyedik algoritmussal fogom kiegészíteni, az adaptív kúpkövetés algoritmusával és annak a változataival. Az előjeles távolságfüggvények hagyományos gömbkövetési technikákkal történő megjelenítése meglehetősen költséges, ezért komplikáltabb jelenetek megjelenítésére nem igazán szokták használni. A szakdolgozatom főbb célja a távolságfüggvények megjelenítésének az optimalizálása.

\section{Dolgozat áttekintése}

Ebben a dokumentumban először a felhasználói dokumentációt mutatom be, azaz a program telepítését és használatát írom le. A(z) \ref{ch:theory}. fejezetben a különböző algoritmusokat fogom részletezni, különös figyelmet fordítva a kúpkövetésre. Ezt követően a fejlesztői dokumentációt mutatom be, a program felépítését és a belső működését. A(z) \ref{ch:testing}. fejezetben a mért eredményeket és a belőlük levont megfigyeléseimet fogom megadni, valamint a program tesztelése is itt lesz részletezve.
