\chapter{Felhasználói dokumentáció}
\label{ch:user_docs}

\section{Rendszerkövetelmények}

A program futtatásához szükség van egy Linux asztali operációs rendszerre, például Linux Mint-re (ami egy Debian alapú disztribúció), Windows alatt nem lett tesztelve, de a C++ kód platformfüggetlen módon lett fejlesztve, ezenkívül erősen ajánlott egy relatív modern dedikált videókártya és az ahhoz tartozó driverek is. A használt GLSL shader verziója (4.30) miatt a rendszernek legalább OpenGL 4.3 verzió kell, hogy fusson.

\section{Telepítés}
\label{se:installation}

A projekthez előre elkészített binárisok nem tartoznak, így, ahhoz, hogy futtatni tudjuk, először le kell fordítani. A projekt a program csomagjainak kezelésére conan-t használ és magát a projektet meson keretrendszerrel lehet építeni, ezeket a programokat pip segítségével szerezhetjük be a PyPI-ről (Python Package Index). Az alábbi útmutató feltételezi, hogy már telepítve van nekünk python, conan, meson, egy alap profil létre lett hozva conan-hoz, a fordításhoz szükséges környezet létrehozásához a projektben található egy segéd bash szkript: \texttt{scripts/setup.sh} és a fordításához és futtatásához pedig egy \texttt{scripts/run.sh} szkript is.

\lstset{caption={}, label=src:bash1}
\begin{lstlisting}[language=bash]
mkdir builddir
conan install . --output-folder=builddir --build=missing
\end{lstlisting}

grafikus könyvtárakat  a conan nem mindig tud rendesen telepíteni, vagy csak részlegesen, ezért szükségesek lehetnek a \texttt{-c tools.system.package\_manager:mode=install -c tools.system.package\_manager:sudo=True} opciók is, de bizonyos használt könyvtárakat (opengl, glu, glew) érdemes lehet először a rendszer csomagkezelőjével (Linux Mint esetében \texttt{apt}) telepíteni

\lstset{caption={}, label=src:bash2}
\begin{lstlisting}[language=bash]
conan install . --output-folder=builddir --build=missing -c tools.system.package_manager:mode=install -c tools.system.package_manager:sudo=True
cd builddir
meson setup --wipe --warnlevel 3 --native-file conan_meson_native.ini .. meson-src
\end{lstlisting}

A program fordítása:

\lstset{caption={}, label=src:bash3}
\begin{lstlisting}[language=bash]
meson compile -C meson-src
meson install -C meson-src --skip-subprojects --quiet
\end{lstlisting}

Végül a program futtatása:

\lstset{caption={}, label=src:bash4}
\begin{lstlisting}[language=bash]
cd meson-src/app
./main
\end{lstlisting}

(a könyvtárváltás fontos, \texttt{./meson-src/app/main} nem elég, mert számít az aktuális munkakönyvtár)
Ha minden jól ment, az alábbi ábrához hasonló a program indítás utáni állapota.

\begin{figure}[H]
	\centering
	\includegraphics[width=0.9\textwidth,keepaspectratio]{gui/normal_view.png}
	\caption{A program indítás utáni állapota. Egy Newton ingája SDF-et láthatunk a képen.}
\end{figure}

\section{Billentyűzet és egér műveletek}

Billentyű műveletek:

\begin{itemize}
	\item \textbf{Escape:} kilépés, applikáció bezárása
	\item \textbf{i:} beállítások ablak megjelenítése, elrejtése
	\item \textbf{w, a, s, d, e, q:} rendre a kamera előre, balra, hátra, jobbra, felfelé, lefelé történő mozgatása
\end{itemize}

Ha a bal egérgombot lenyomva tartjuk és mozgatjuk az egeret, akkor az a kamerát fogja mozgatni, ha pedig a jobbat tartjuk lenyomva, akkor pedig a mozgatás hatására nagyítani/kicsinyíteni fogja a képet. A görgő segítségével is (jobb egér gomb + mozgatáson kívül) lehet zoomolni.

\section{Beállítások}

Az ablak több lenyíló fülre van osztva, amelyekre ha rákattintunk az egérrel akkor az adott almenü lenyílik, némelyik menü pont további részekre is van osztva.

\begin{figure}[H]
	\centering
	\includegraphics[width=0.9\textwidth,keepaspectratio]{gui/gui_1.png}
	\caption{A beállítások menü megnyitás utáni állapota látható.}
\end{figure}

\begin{figure}[H]
	\centering
	\includegraphics[width=0.9\textwidth,keepaspectratio]{gui/gui_6.png}
	\caption{A beállítások menü első 4 almenüje látható kibontva.}
\end{figure}

Ezek közül az első az SDF jelenet kiválasztására ad lehetőséget, például ha kiválasztjuk a „primitives” opciót, akkor (a menüt elrejtve) valahogy így fog kinézni:

\begin{figure}[H]
	\centering
	\includegraphics[width=0.9\textwidth,keepaspectratio]{gui/sdf2_zoomed.png}
	\caption{Különböző SDF primitívekből álló jelenet látható.}
\end{figure}

A második opció a különböző nézeti módok között ad választhatósági lehetőséget, lásd a(z) \ref{se:rendering_modes}. fejezetet.

A harmadik és negyedik fül a különböző módoktól független paramétereket tartalmazza, ezek közül a harmadik a gömb/kúpkövetés paraméterei, míg a negyedik opciói az árnyékok, ambiens kitakarás és a maximum tükröződések állítására szolgálnak.

A következő (5. fül) almenüben a használt módszert kiválasztó rádiógombok és a relaxált valamint a javított módszerhez tartozó gömbkövetési paraméterek (a naiv módszerhez külön nincsenek) találhatóak.

\begin{figure}[H]
	\centering
	\includegraphics[width=0.9\textwidth,keepaspectratio]{gui/gui_10.png}
	\caption{5. almenü részlegesen kibontva látható.}
\end{figure}

Az 5. almenü még a kúpkövetéshez tartozó paramétereket is tartalmazza, ezekkel a kiindulási kúpméret állítható, a köztes epszilon, köztes (kúpkövetés alatti) gömbkövetési módszer (módosított, hogy működjön a kúpkövetéssel) és az azokhoz tartozó paraméterek, valamint a kúpkövetés befejezte után futó gömbkövetési módszer és az azokhoz tartozó paraméterek is

\begin{figure}[H]
	\centering
	\includegraphics[width=0.9\textwidth,keepaspectratio]{gui/gui_11.png}
	\caption{5. almenü, a kúpkövetéshez tartozó beállítások láthatóak.}
\end{figure}

Az utolsó menüpontban pedig a „benchmark”-hoz tartozó paraméterek: a viszonyítási minta iteráció limitje, valamint azok a paraméterek, amelyek meghatározzák, hogy milyen iteráció limitek mellett fusson le és hányszor. Ezen kívül még egy gomb is található, amely elindítja a „benchmark”-ot, itt megjegyzendő, hogy a „benchmark” futása alatt, az ablakot bezárni és rendesen interaktálni vele nem lehet, hiszen szinkronizált módon lett implementálva, így blokkol, amíg be nem fejeződik.

\begin{figure}[H]
	\centering
	\includegraphics[width=0.9\textwidth,keepaspectratio]{gui/gui_12.png}
	\caption{6. almenü, a tesztelés paraméterei láthatóak.}
\end{figure}






















